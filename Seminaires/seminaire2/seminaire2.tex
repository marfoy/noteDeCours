\documentclass[10pt,a4paper]{article}
\usepackage[utf8]{inputenc}
\usepackage[T1]{fontenc}
\usepackage{amsmath}
\usepackage{amsfonts}
\usepackage{amssymb}
\usepackage{graphicx}
\title{Seminire 2 : Les trois piliers du référencement sur le Web}
\author{Martin Lempereur}
\begin{document}
\maketitle
\section{Intro : Imagine}
Dr Ir Robert Viseur alias Jonnhy.
Mr Lenon a commencer le référencement en 2005, les fait plein de truc pour le fun.
Trois tempiliers, premier référencement naturel
deuxieme référencement commercial
troisieme référencement social,
Mr Lenon est chercheur au cetic à moitié sinon il assiste en tant qu'assistant à l'unif de Mons,
il kiff l'open source, l'évaluation de techno et le traitement de l'information.
Il fait aussi des photo avec Yokohono.
Les référencement c'est moettre des pages web dans des moteur de recherche en veillant a la position, (pas la PLS ci possible), Sur google les pubs de merde c'est des lien commerciaux(ref comm) au dessus les reférencement naturel. L'importance de se mettre bien ? Les user il on un strabisme qui tend vers le coin superieur gauche. Après 20 pages les gens commence a s'enmerde a chercher.
Depuis que google a rachéter son petit toutou de youtube il y a des info qui ne sont plus textuelles style
des videos, Les autres moteurs ont chié dans le froc et se partagent a eux tous 10\% d part du marché pour l'europe.
Y a des pays ou google il en chie un peu plus style en Chine(26\%), Corée du Sud(3\%).
Aux USA il ont 63\%.
Appli sous Flash et Ajax sont plus chaud a référencer comme le doublage statique (pour les SPA)

Un moteur de recherche c'est un automate à trois partie:
-Crawler extraction de liens et explore ses liens. D'ou le prob avec Flash il ne peut pas suivre les liens le pauvre nono le robot.
-Un indexeur, creation de dictionnaire inversé chaque mot associé a un ensemble de pages
-Une interface d'intérrogation.

Un site avec des news aura plus de chance d'apparaitre dans les recherche d'actualité.
Réfréencement via fromulaire ou apparition sur un autre site déja référencé.

On peut configurer un robot.txt pour dire ce que l'on veut ou non référencer

Le texte d'un lien est imortant pour l'indexation !

\section{Premier tempiliers du référencement}
\section{Deuxième tempiliers du référencement}
\section{Troisième tempiliers du référencement}
\end{document}