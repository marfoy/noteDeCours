\documentclass[10pt,a4paper]{article}
\usepackage[utf8]{inputenc}
\usepackage[francais]{babel}
\usepackage[T1]{fontenc}
\usepackage{amsmath}
\usepackage{amsfonts}
\usepackage{amssymb}
\usepackage{graphicx}
\author{Giorgio le Magnifique}
\title{Semih Nerf 2 : Les trois piliers du référencement sur le web}
\begin{document}
\maketitle
\textit{Slides réalisés par \'Eveline Daubie}
\section{Présentation de Robert VISEUR}
Robert Viseur est le sosie officiel belge de John Lennon. 
Il a servi de conseil technique pour Google en 2006. Travaille au CETIC et assistant à la Polytech (mais il est sous xubuntu donc ça va).  Site : www.voipfr.org ; Photographe à ses heures perdues.\\ Est très joueur.
\begin{figure}[h!]
\centering
\includegraphics[scale=0.4]{lennon.jpg}
\caption{John Lennon}
\end{figure}
\section{Les piliers}
\subsection{Premier pilier : référencement naturel}
présentation acceptable pour le moteur de recherche.
prendre le contenu d'un site web et le présenter au moteur de recherche avec un objectif de visibilité (avoir un meilleur placement dans le moteur de recherche, la meilleure position dans l'ensemble des résultats fourni).\\
Google : 90\% de parts du marché. Google Chine : 27\%, Corée du Sud : 3\% \\
\textbf{Formats de données sur Internet ?} : 
\begin{itemize}
\item HTML sous plusieurs versions et relativement simples à référencer.
\item Flash (tendance à disparaitre)
\item Ajax (ex: Facebook)
\end{itemize}
\textbf{Moteur de recherche ?}
\begin{figure}[h!]
\centering
\includegraphics[scale=0.5]{referencement.png}
\caption{Exemple de bon référencement. \textit{Ce polytechnicien serait-il une putaclique ?}}
\end{figure}
Un crawler est un bot qui extrait les liens des sites et le suit pour explorer l'internet.\\
Problème : javascript :> le bot ne trouve pas les liens :> difficilement référenciable par Google.\\
Fonction d'indexeur : analyse les pages pour en extraire le texte et le restructure pour le rendre cherchable.\\
Ex de référencement : utiliser des images bien référencées :> etre + visible sur google actus ou images.\\
Google : formulaire d'ajout. Un peu inutile car google le trouve automatiquement et récursivement si on met le lien sur un site déjà référencé. \\
robots.txt => fichier permettant d'indiquer au robot ce qu'il peut faire ou ne pas faire sur notre site.
\\Google bombing : GB était encore président des états unis. Les gens ont mis dans leurs liens hypertexte "lamentable échec" avec un lien vers le site de la maison blanche. \\
Le webmaster essaie de trouver des failles pour optimiser sa visibilité et Google les contre avec des filtres. (épée <-> bouclier)
\begin{figure}[h!]
\centering
\includegraphics[scale=0.3]{webmaster.png}
\caption{webmaster officiel de l'age sciences}
\end{figure}
\paragraph{Exploiter les liens externes et internes}
Chaque page est boostée par des "citations" :> hyperliens ou lien hypertexte; citation interne et citation externe. Google a introduit le trustrank : l'indice de confiance.
\paragraph{Réécriture d'URL}
But : de passer de la notation normale prévue par les langages de programmation par des index à des champs plus basiques et idéalement d'avoir des URL avec les mots clés qui nous intéresse dans l'URL. 

\subsection{Deuxième pilier : référencement commercial}
utilisation d'outils commerciaux
\paragraph{Referencer sur google images}
nom de fichier explicite exemple : RobertViseur.jpg et pas xXR0b3rtVis3uRsnip4rzXx.jpg \\

\subsection{Troisième pilier : référencement social}
utiliser les réseaux sociaux
\end{document}