\documentclass[a4paper,10pt]{report}
\usepackage[utf8]{inputenc}
\usepackage{listings} % http://ctan.org/pkg/listings
\lstset{
  basicstyle=\ttfamily,
  mathescape
}

% Title Page
\title{Compilation}
\author{Martin Lempereur}


\begin{document}
\maketitle
\tableofcontents




  \chapter{Introduction}

Le compilateur peut être vu comme une sorte de "traducteur".
Il prend en entrée un programme source et le traduit en un autre programme dans un langage différent
que l'on appelle programme cible, souvent le langage du programme cible est plus bas niveau que le langage source par exemple: langage machine, langage assembleur.


Les étapes principales du compilateur:


Programme source $=>$ [*[ analyseur lexical $=>$ analyseur syntaxique $=>$ analyseur sémantique ] $=>$ **[générateur de code intermédiare $=>$ optimisation du code $=>$ génération du code ]] $=>$ programme cible

*représente la partie analyse du compilateur, celle que l'on verra durant le cours


**représente la partie synthèse du compilateur, non vue au cours

En parallèle avec ces étapes il y a deux étapes supplémentaire qui sont executée, celle de la gestion des erreurs et celle du gestionnaire de la table de symbole.

\section{Illustration par un exemple}

Soit l'instruction suivante,
\begin{lstlisting}
position := initial + rate * 60;

 
\end{lstlisting}


\subsection{Analyseur lexical}

l'analyseur lexical reçoit en entrée un programme source vu comme une suite de symboles tapés au clavier.


\subsubsection{But}
\begin{itemize}
 \item écarter les commentaires et caractères blancs
 \item détécter les \textbf{lèxemes} formant le programme
        
\end{itemize}


Ex:


\begin{description}
 \item[position :] lexème qui est un identificateur
 \item[:= :] lexème qui est une affectation
 \item[initial :] lexème qui est un identificateur
 \item[+ :] lexème qui est un opérateur
 \item[rate :] lexème qui est un identificateur
 \item[* :] lexème qui est un opérateur
 \item[60 :] lexème qui est une constante entière
 \item[; :] lexème qui est un séparateur
\end{description}                

séquence en sortie de l'analyseur lexical :

[position][:=][initial][+][rate][*][60][;]

le compilateur peut voir la suite de symboles comment ceci : 

ID := ID + ID * NBR
\subsubsection{Cours}
Outils théorique pour la gestion des lexèmes
\begin{itemize}
 \item Expression régulière (regex)
 \item Automates fini
\end{itemize}

\subsubsection{En plus}
  \begin{itemize}
   \item Detection d'erreurs lexicales
   \item Tout identificteur trouvé va être stocké dans \textbf{la table des symboles}
  \end{itemize}
  
 \subsection{Analyseur Syntaxique}
  \subsubsection{But}
  Vérifier que le programme source respecte la \textbf{syntaxe} du langage de programmation utilisée
  \begin{itemize}
   \item En utilisant la liste de lexèmes générée par l'analyseur lexical
   \item En produisant en sortie un \textbf{arbre de dérivation} (voir schéma cours page 2)
  \end{itemize}
  \subsubsection{Cours}
  Outils théoriques
  \begin{itemize}
   \item Grammaires hors contact
  \end{itemize}
    \subsubsection{But}
  \begin{itemize}
   \item detection d'erreur syntaxique, ex: oubli d'un ;
   \item ajout de certaines \textbf{caractéritiques} aux identificateurs stockés dans la table des symboles
   ex : les 3 ID sont des identificateurs de \textbf{variables}
  \end{itemize}
  
  \subsection{Analyseur sémantique}
  \subsubsection{But}
    \begin{itemize}
     \item \underline{Contrôle de type:} donner un type à chaque variable et vérifier qu'il n'y ai pas d'erreur de type
     \item transformer l'arbre de dérivation reçu de l'analyseur syntaxique en un arbre plus compact appelé \textbf{arbre abstrait}
    \end{itemize}
    
    \subsection{Générateur de code intermédiaire}
      \subsubsection{But}
      Sur base de l'arbre abstrait traduction de chaque instruction du programme source en une série d'instruction simple (ici traduction en assembleur)
      
      \begin{lstlisting}
      $ID_{1}$ := $ID_{2}$ + $ID_{3}$ * 60  
      \end{lstlisting}
      
      traduit par:
      
      \begin{lstlisting}
       TEMP1 = INT_REAL(60)
       TEMP2 = $ID_{3}$ * TEMP1
       TEMP3 = $ID_{2}$ + TEMP2
       $ID_{1}$ = TEMP3
       
      \end{lstlisting}


      \subsection{Optimisation du code intermédiaire}
      \subsubsection{But}
      Réecrire de façon optimisée la liste des instructions générées à l'étape précédente.
      
      Ex:
      
      Le code précédent devient :
      
      \begin{lstlisting}
       TEMP1 = $ID_{3}$ * 60
       $ID_{1}$ = $ID_{2}$ + TEMP1
      \end{lstlisting}

    \subsection{Génération du code final}
    \subsubsection{But}
    Générer le programme cible
    
    \underline{Ici en assembleur :}
    
    \begin{lstlisting}
     MOVF $ID_3$, R2
     MULF #60.0, R2
     MOVF $ID_2$, R1
     ADDF R2, R1
     MOVF R1, $ID_1$
    \end{lstlisting}
    
    R1, R2 deux registres et F pour flottant dans MULF, MOVF et ADDF
    
    \subsection{Gestionnaire des erreurs}
    \begin{itemize}
     \item Détecter les erreurs dans le programme
     \item Annoncer clairement les erreurs
     \item Poursuivre correctement la compilation après chaque erreurs détectées
    \end{itemize}
 \subsection{Gestionnaire de la table des symboles}
    \begin{itemize}
     \item On stocke dans une table tous les identificateurs détectés avec leurs caractéristiques
     
     ex: 
     
     position : identificateur de variable réelle
     
     \item Cette table est construite pendant l'analyse et elle est utilisée pendant la synthèse
         
    \end{itemize}
    
    \chapter{Analyse lexicale}
    
    programme source $=>$ [Analyseur lexical] $=>$ liste de lexèmes
    
    En parallèle a lieu des intéractions avec le gestion d'erreur et le gestionnaire de la table de symbole.
    
    \section{But}
    \begin{itemize}
     \item Lire un à un les symboles du programme source
     \item Au fur a mesure de cette lecture, détecter les lexèmes
     \item Ecarter les blancs et les commentaires
     \item Detecter et gérer les erreures lexicales
     \item Stocker dans la table des symboles tous les identificateurs du programme source
    \end{itemize}
    
    \section{Exemple en JAVA}
    
      \underline{Types de lexèmes}
      \begin{enumerate}
       \item Blancs
       \item Commentaires
       \item Identificateurs
       \item mots-clés
       \item littéraux
       \item séparateurs
       \item Opérateurs
      \end{enumerate}
      
      \subsection{Blancs}
      On condidère comme blancs les espaces , tabulation $=>$ et retour à la ligne $/n$
      
      \subsection{Commentaires}
      
      //commentaire uniligne
      
      /* commentaire multi-lignes */
      
      \subsection{Identificateurs}
      
      Il faut une définition précise sur laquelle se basé, en Java cette définition est:
      
      ``An identifier is an unlimited-length sequence of Java letters and Java digits, the first of which must be a Java letter.''
      
      Lettre JAVA: a-z, A-Z,\textunderscore,\textdollar
      
      Chiffres JAVA: 0-9
      
      \subsection{Mots-clé}
      
      Il y en a 50 en JAVA.
      
      Ex: abstract,while,void,...
      
      \subsection{Littéraux}
      \begin{itemize}
       \item Entier
       \item Reel
       \item Booleens
       \item caractères
       \item Chaines de caractères
       \item Null
      \end{itemize}
      
      \subsubsection{Littéraux entier}
      \underline{Base 10 :}
      Definition précise :
      
      ``A décimal numéral is either 0 or one digit among 1-9 optionally followed by one or more digits''
      
      
      \underline{Base 8 :}
      
      Longueur $>=$ 2
      \newline $1^{er}$ digit 0 suivi par 0-7
      
      \underline{Base 16 :}
      
      Commence par 0x ou 0X suivi par un ou plusieurs digit parmi 0-9 ou A-F
      
      zéro en base 10: 0
      
      zéro en base 8: 00
      
      zéro en base 16: 0x0
      
      \subsection{Séparateurs}
      
      Un séparateur en Java est un des symboles suivant: \textbf{( ) { } [ ] ; , .}
      
      \subsection{Opérateurs}
      
      Le langage Java compte les opérateurs suivante dans ses symboles : $=$ $<$ \& ...
      
      \subsection{Lexèmes}
      
      \underline{Définition précise du type de lexèmes} via les \underline{expressions régulières}.
      
      
      Mais d'abord un peu de vocabulaire:
      
      \begin{description}
       \item[alphabet] ensemble fini de \underline{symboles}
       \item[mots] suite finie, evnetuellement vide de symboles pris dans un alphabet
       \item[mots vide] mot utilisant 0 symbole noté $\varepsilon$
       \item[langage] ensemble de mots sur un alphabet donné
      \end{description}

      









\end{document}          
